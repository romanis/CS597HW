\documentclass{article}
% Change "article" to "report" to get rid of page number on title page
\usepackage{amsmath,amsfonts,amsthm,amssymb}
\usepackage{setspace}
\usepackage{Tabbing}
\usepackage{fancyhdr}
\usepackage{lastpage}
\usepackage{extramarks}
\usepackage{url}
\usepackage{chngpage}
\usepackage{longtable}
%\usepackage{subfigure}
\usepackage{soul,color}
\usepackage{graphicx,float,wrapfig}
%\usepackage{caption,subcaption}
\usepackage{enumitem}
\usepackage{morefloats}
\usepackage{multirow}
\usepackage{multicol}
\usepackage{indentfirst}
\usepackage{lscape}
\usepackage{pdflscape}
\usepackage{natbib}
%\usepackage[toc,page]{appendix}
\providecommand{\e}[1]{\ensuremath{\times 10^{#1} \times}}

% In case you need to adjust margins:
%\topmargin=-0.45in      % Switch to the other top for overleaf
\topmargin=0.25in      %
\evensidemargin=0in     %
\oddsidemargin=0in      %
\textwidth=6.5in        %
%\textheight=9.75in       % play with this for overleaf
\textheight=9.25in       %
\headsep=0.25in         %

% Homework Specific Information
\newcommand{\hmwkTitle}{Introductions}
\newcommand{\hmwkDueDate}{Monday,\ August\  27,\ 2018}
\newcommand{\hmwkClass}{Homework 0}
\newcommand{\hmwkClassTime}{CSE 597}
\newcommand{\hmwkClassInstructor}{ } 
\newcommand{\hmwkAuthorNameb}{Roman Istomin (rji5040@psu.edu)}
\newcommand{\hmwkNames}{HW0}

% Setup the header and footer
\pagestyle{fancy}
\lhead{\hmwkNames}
\rhead{\hmwkClass: \hmwkTitle} 
\cfoot{Page\ \thepage\ of\ \pageref{LastPage}}
\renewcommand\headrulewidth{0.4pt}
\renewcommand\footrulewidth{0.4pt}




%%%%%%%%%%%%%%%%%%%%%%%%%%%%%%%%%%%%%%%%%%%%%%%%%%%%%%%%%%%%%
% Make title
\title{\vspace{2in}\textmd{\textbf{\hmwkClass:\ \hmwkTitle}}\\\normalsize\vspace{0.1in}\small{\hmwkDueDate}\\\vspace{0.1in}\large{\textit{\hmwkClassInstructor\ \hmwkClassTime}}\vspace{3in}}
\date{}
\author{\textbf{\hmwkAuthorNameb} } % \\ \textbf{\hmwkAuthorNamea}}
%%%%%%%%%%%%%%%%%%%%%%%%%%%%%%%%%%%%%%%%%%%%%%%%%%%%%%%%%%%%%

\begin{document}
\begin{spacing}{1.1}
\maketitle

\newpage
\section{Syllabus Acknowledgement}

By turning in this assignment, I, \hmwkAuthorNameb, acknowledge that I have received and understand the course syllabus information available on \url{sites.psu.edu/psucse597fall2018}. 

\section{Introduction}

My name is \hmwkAuthorNameb.  I am a 5th year PhD student in the Economics department. My programming experience includes C++, Matlab, Python and shared memory parallelization methods (mostly OpenMP).  When I compute, I typically use ACI-B.  My research is mostly computational in nature. 

My area of interest is computational Industrial Organization and Education research. For my job market paper, I am implementing non-parametric estimator first envisioned in \cite{agarwal2018demand}. I use Gurobi linear solver to check feasibility of group system of linear inequalities. 

I also work on numerical inversion of Pure Characteristics model of demand proposed by \cite{berry2007pure}. To do that I compute the jacobian of direct mapping from structural parameters to market shares and use it in conjunction with interior point optimizer by Knitro 10.3 to find inverse mapping. 


\subsection{Accounts}

I have gotten an account on ACI using \url{https://ics.psu.edu/?page_id=57}.  My ACI username is rji5040@psu.edu.

I have gotten an account on XSEDE using \url{https://portal.xsede.org/my-xsede?p_p_id=58&p_p_lifecycle=0&p_p_state=maximized&p_p_mode=view&saveLastPath=0&_58_struts_action}.  My username is rom4ik (or rji5040@psu.edu, I use both to sign in).

I will be making my assignments available using Github (https://github.com/romanis/CS597HW). 

\subsection{My Course Project}

I am currently thinking about choosing least squares problem or the eigen value decomposition as my $Ax=b$ problem for the semester project. I believe that this will be a good project because
\begin{itemize}
  \item Least squares is widely used in Economics
  \item Eigen value decomposition is widely used in optimization
\end{itemize}


\section{HW 0 Code and Writeup}

You can get my assignment onto ACI using the command:

\begin{verbatim}
git clone git@github.com:romanis/CS597HW.git
\end{verbatim}



\subsection{Program overview}

This is a simple hello world program, written in C++. There is only one code file. The repository also contains the makefile for creating the executable, a readme, licensing information and the tex file for the write-up.


\subsection{Instructions for running and verifying the code}

\textbf{Creating the executable:}
\begin{verbatim}
module load gcc/7.3.1
make
\end{verbatim}

\textbf{Running the program:}
\begin{verbatim}
./hello
\end{verbatim}

\textbf{Expected output:}
\begin{verbatim}
What is your name?
Roman
hello, Roman
\end{verbatim}

\subsection{Instructions for compiling the write-up}

I used ACI to compile the document.  You can do this using the command:
\begin{verbatim}
./pdfmake.sh
\end{verbatim}

\section{Acknowledgements}

I would like to acknowledge Chris Blanton and Chuck Pavloski for helping formulate the homework material.



\end{spacing}
\bibliographystyle{apalike}
\bibliography{schools}

\end{document}

%%%%%%%%%%%%%%%%%%%%%%%%%%%%%%%%%%%%%%%%%%%%%%%%%%%%%%%%%%%%%}}
